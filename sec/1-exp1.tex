\fancyhead[LH]{传感器综合实验课程报告}
\fancyhead[RH]{实验一\quad 电阻应变片式传感器实验}
\section{电阻应变片式传感器实验}

\begin{center}
    {\zihao{-4}
        小组成员:\underline{\quad 闵鹏远、王安彤 \quad} \quad
        实验时间:\underline{\quad 2025年3月28日 \quad}
    }
\end{center}

\subsection{应变片双臂特性试验}

\subsubsection{实验目的}

\begin{enumerate}
    \item 了解金属箔式应变片的工作原理与应变效应
    \item 掌握应变片式电阻传感器的
    \item 掌握机械回程差的消除方法。
\end{enumerate}

\subsubsection{实验内容}

针对金属箔式应变片,构建半桥电桥电路,如下图,将电阻应变式传感器的电阻变化转换成电压或电流信号。通过调节测微头微分筒改变梁的受力程度,测试电阻应变片的特性。

\subsubsection{实验步骤}

\subsubsection{实验结果}

1. 原始数据

2. 曲线图

3. 灵敏度

4. 非线性误差

\subsection{应变片温度特性试验及温度补偿实验}

\subsubsection{实验目的}

\subsubsection{实验内容}

\subsubsection{实验步骤}

\subsubsection{实验结果}

1. 原始数据

2. 计算

3. 分析比较

\subsection{应变片直流全桥的应用——电子秤实验}

\subsubsection{实验记录}

\subsubsection{电压与重量的关系公式}

\subsubsection{实验结果分析}

\subsubsection{称重两个物品重量的计算}

\subsection{测试样例}

这里用于测试内容,请后续直接删除即可 

图片引用测试\ref{fig:myfig2}:

\begin{figure}[H] % 这里 H 表示 强制放置在当前位置,更改为htbp表示根据需要放置
    \centering
    \includegraphics[width=0.8\textwidth]{fig/北京理工大学校徽.png} % 可调整宽度
    \caption{测试图片}
    \label{fig:myfig2}
\end{figure}

公式示例\ref{eq:pid-incremental}:
\begin{equation}
    \left\{
    \begin{aligned}
        u(k) &= q_0 e(k) + B(k-1) \\
        B(k) &= u(k) + q_1 e(k-1) + q_2 e(k-2) \\
        q_0 &= K_p \left(1 + \frac{T_s}{T_i} + \frac{T_d}{T_s} \right) \\
        q_1 &= -K_p \left(1 + 2 \cdot \frac{T_d}{T_s} \right) \\
        q_2 &= K_p \cdot \frac{T_d}{T_s}
    \end{aligned}
    \right.
    \label{eq:pid-incremental}
    \end{equation} 

表格示例:
\input{tab/tab1.tex} % 这里的 tab/tab1.tex 是表格文件的路径

\newpage